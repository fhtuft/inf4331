\documentclass[a4paper]{article}

% Import some useful packages
\usepackage[margin=0.5in]{geometry} % narrow margins
\usepackage[utf8]{inputenc}
\usepackage[english]{babel}
\usepackage{hyperref}
\usepackage{minted}
\usepackage{amsmath}
\usepackage{xcolor}
\definecolor{LightGray}{gray}{0.95}

\title{Peer-review of assignment 5 for \textit{INF3331-fhtuft-master}}
\author{Reviewer 1, krimha, {mail@uio.no} \\
 		Reviewer 2, pierreyp, {mail@uio.no} \\
		Reviewer 3, tnbildst, {mail@uio.no}}

\begin{document}
\maketitle

\section{Introduction}
\subsection{Goal}
The review should provide feedback on the solution to the student. The main goal is to \emph{give constructive feedback and advice} on how to improve the solution. You, the peer-review team, can decide how you organise the peer-review work between you. 

\subsection{Guidelines}\label{sec:general_review}
For each (coding) exercise, one should review the following points:

\begin{itemize}
  \item Is the code \textbf{working as expected}? For non-internal functions (in particular for scripts that are run from the command-line), does the program handle invalid inputs sensibly?
  \item Is the code \textbf{well documented}? Are there docstrings and are the useful?
  \item Is the code written in \textbf{Pythonic way} \footnote{https://www.python.org/dev/peps/pep-0020/}? Is the code easy to read? Are the variable/class/function names sensible? Do you find overuse of classes or not sufficient use of functions or classes? Are there parts of the program that are hard to understand? 
  \item Can you find \textbf{unnecessarily complicated parts} of the program? If so, suggest an improved implementation.
  \item List the programming parts that are not answered.
\end{itemize}
Use (shortened) code snippets where appropriate to show how to improve the solution. 

\subsection{Points}
The review is completed by pushing the review Latex and PDF files to each of the reviewed repositories. The name of the files should be \emph{feedback.tex} and \emph{feedback.pdf}.

You will get up to 10 points for delivering the peer-reviews. Each of you should contribute to the review roughly equivalently - your team will get the same number of points\footnote{In case a team-member does not contribute, please email \href{mailto:simon@simula.no}{simon@simula.no}}. 

\subsection{Useful Latex snippets}
Here are some sample usage of Latex.

\subsubsection{Sample code}
\begin{minted}[bgcolor=LightGray, linenos, fontsize=\footnotesize]{python}
import sys
print "This is a sample code"
sys.exit(0)
\end{minted}

\subsubsection{Mathematical equation}
\begin{align}
2 \pi > 6
\end{align}



\section{Review \emph{- to be filled out}}\label{sec:review}

System (Python version 2.7, Windows 7) 

%%%%%%%%%%%%%%%%%%%%%%%%%%%%%%%%%%%%%%%%%%%%%%%%%%%%%%%%%%
\subsection*{General feedback}
Everything looks great and worked fine. Great job! There wasn't really alot for me to comment on, as I found everything to be working perfectly. 

%%%%%%%%%%%%%%%%%%%%%%%%%%%%%%%%%%%%%%%%%%%%%%%%%%%%%%%%%%
\subsection*{Assignment 5.1: Syntax highlighting}
\begin{itemize}
\item The code seems to be working fine, with handling for no arguments as well, which is nice. 
\item I like that you've also been commenting the code nicely, and following the "pythonic" way of writing code. 
\end{itemize}


%%%%%%%%%%%%%%%%%%%%%%%%%%%%%%%%%%%%%%%%%%%%%%%%%%%%%%%%%%
\subsection*{Assignment 5.2: Python syntax} \label{sec:assignment5.2}
\begin{itemize}
\item Nothing to comment on really. The python syntax worked and looks quite nice. 
\end{itemize}

%%%%%%%%%%%%%%%%%%%%%%%%%%%%%%%%%%%%%%%%%%%%%%%%%%%%%%%%%%
\subsection*{Assignment 5.3: Syntax for your favorite language}
Language: C
\begin{itemize}
\item Many variables implemented in the theme, which is cool.
\item The theme works and looks nice - nothing to comment on. 
\end{itemize}


%%%%%%%%%%%%%%%%%%%%%%%%%%%%%%%%%%%%%%%%%%%%%%%%%%%%%%%%%%
\subsection*{Assignment 5.4: Syntax for your second favorite language}
Language: Java
\begin{itemize}
\item Many variables implemented in the theme, which is cool.
\item The theme works, looks nice and is different from the one in task 5.3 - nothing to comment on.  
\end{itemize}

%%%%%%%%%%%%%%%%%%%%%%%%%%%%%%%%%%%%%%%%%%%%%%%%%%%%%%%%%%
\subsection*{Assignment 5.5: superdiff}
\begin{itemize}
\item The program runs as intended, with no errors. 
\item I like how you distinct between additions, deletion, permutation etc. in the print, making it easy to read.
\item Good commenting in mydiff. The code is easily read, and follows the "pythonic" way of writing code, which is nice. 
\end{itemize}

%%%%%%%%%%%%%%%%%%%%%%%%%%%%%%%%%%%%%%%%%%%%%%%%%%%%%%%%%%
\subsection*{Assignment 5.6:  Coloring diff}
\begin{itemize}
\item The highlighting is correct, as stated in the task - with greens for additions, red for deletion and the no coloring. The code runs with no errors. 
\item Again, great distinctions in the print.
\end{itemize}
 


\bibliographystyle{plain}
\bibliography{literature}

\end{document}